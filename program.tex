
\documentclass[10pt]{article}
% no page numbers
\pagenumbering{gobble}
% define page size
% height 19cm
% width 10cm
\usepackage{geometry}
%\geometry{papersize={10cm,19cm},total={8.5cm,17cm},top=1.5cm,left=1.8cm}
\geometry{paperheight=19cm, paperwidth=10cm,total={6.5cm,17cm},top=0.5cm,left=1.6cm}

\usepackage[UTF8]{ctex}

\usepackage{etoolbox}

\usepackage{setspace}
%\setstretch{1.12}


% \usepackage{graphicx}
% \usepackage{tikz}
\usepackage{eso-pic,graphicx}


  % \AddToHookNext{shipout/background}{
  %   \begin{tikzpicture}[remember picture,overlay]
  %   \node at (current page.center) {
  %     \includegraphics[width=\paperwidth]{background.jpg}
  %   };
  %   \end{tikzpicture}
  % }
  


% this is a page of a concert program
% it consists of a list of pieces
% each piece has a Chinese title, with bigger font, all other text is smaller
% an English title, the composer's chinse and english name, players Chinese names and players Englishe names, arranger's name (optional), 
% define a new command to hold the piece

\newcommand{\piece}[8]{
    % set #1 to a bigger font
    % Chinese title
    \fangsong
        \noindent{\\\normalsize #1}\\
        \scriptsize
    % english title
        #2  \ifstrempty{#8}{}{\vspace{-0.1cm}}\\
        \ifstrempty{#8}{}{{\tiny #8 \\}}
    % composer chinese and english name
        #3 \quad #4\\
    % arranger (optional)
    \ifstrempty{#7}{}{{\tiny #7 改编}\\}

    % define a two by two tabular
    % @{} is for noindent
% \noindent\begin{tabular}{@{}l p{4.8cm}}
%          \noindent 演奏者 & #5\\
%         Performer & #6\\
%     \end{tabular}
\vspace{-0.2cm}
    \noindent #5\\
    \ifstrempty{#6}{}{{\tiny #6}\\}



}

    




\begin{document}

\AddToShipoutPictureBG{\includegraphics[width=\paperwidth,height=\paperheight]{background.jpg}}

\piece{花之圆舞曲}{Waltz of the Flowers}{柴可夫斯基}{Pyotr Ilyich Tchaikovsky}{卢俊义、于禁}{LU Junyi, Yu Jin}{Eduard Langer}

\piece{叙事曲: 第一首}{Ballade, Op.23: No.1}{肖邦}{Frédéric Chopin}{范家乐}{FAN Jiale}{}

\piece{C大调幻想曲第二乐章}{Frantasie in C major II. Allegro}{舒伯特}{Franz Schubert}{王二小(小提琴)、施耐奄}{WANG Erxiao (Violin), SHI Nai'an}{}

\noindent\begin{center}\normalsize \hspace{-1.5cm} 中场休息 Intermission\end{center}
\vspace{0.3cm}
\piece{八首音乐会练习曲: 田园}{Eight Concerto Étude, Op.40
VI. Pastorale}{卡普斯汀}{Nikolai Kapustin}{范家乐}{FAN Jiale}{}

\piece{降E小调钢琴奏鸣曲}{Piano Sonata in E minor, Op.26
IV. Fuga}{巴伯}{Samuel Barber}{陈雨芊}{CHEN Yuqian}{}






% \thispagestyle{empty}

\end{document}

